%!TEX TS-program = xelatex
\documentclass[]{friggeri-cv}
\usepackage{afterpage}
\usepackage{hyperref}
\usepackage{color}
\usepackage{xcolor}
\usepackage{smartdiagram}
\usepackage{fontspec}
% if you want to add fontawesome package
% you need to compile the tex file with LuaLaTeX
% References:
%   http://texdoc.net/texmf-dist/doc/latex/fontawesome/fontawesome.pdf
%   https://www.ctan.org/tex-archive/fonts/fontawesome?lang=en
%\usepackage{fontawesome}
\usepackage{metalogo}
\usepackage{dtklogos}
\usepackage[utf8]{inputenc}
\usepackage{tikz}
\usetikzlibrary{mindmap,shadows}
\hypersetup{
pdftitle={Alex Irmel Oviedo Solis - Ing. Informatico y de Sistemas},
pdfauthor={Ing. Alex Irmel Oviedo Solis},
pdfsubject={Curriculum vitae},
pdfkeywords={},
colorlinks=false,           % no lik border color
allbordercolors=white       % white border color for all
}
\smartdiagramset{
bubble center node font = \footnotesize,
bubble node font = \footnotesize,
% specifies the minimum size of the bubble center node
bubble center node size = 0.5cm,
%  specifies the minimum size of the bubbles
bubble node size = 0.5cm,
% specifies which is the distance among the bubble center node and the other bubbles
distance center/other bubbles = 0.3cm,
% sets the distance from the text to the border of the bubble center node
distance text center bubble = 0.5cm,
% set center bubble color
bubble center node color = pblue,
% define the list of colors usable in the diagram
set color list = {lightgray, materialcyan, orange, green, materialorange, materialteal, materialamber, materialindigo, materialgreen, materiallime},
% sets the opacity at which the bubbles are shown
bubble fill opacity = 0.6,
% sets the opacity at which the bubble text is shown
bubble text opacity = 0.5,
}

\addbibresource{bibliography.bib}
\RequirePackage{xcolor}
\definecolor{pblue}{HTML}{0395DE}

\begin{document}
\header{Alex Irmel}{Oviedo Solis}
{Ing. Inform\'atico de Sistemas}

% Fake text to add separator
\fcolorbox{white}{gray}{\parbox{\dimexpr\textwidth-2\fboxsep-2\fboxrule}{%
.....
}}

% In the aside, each new line forces a line break
\begin{aside}
    \includegraphics[scale=0.18]{img/aoviedo.png}
    \section{Direcci\'on}
    Av. de la cultura 2505
    Cusco, Cusco
    ~
    \section{Telefonos}
    +51 930 328 402
    ~
    \section{Correo electronico}
    \href{mailto:aoviedosolis@gmail.com}{\textbf{aoviedosolis@}\\gmail.com}
    \href{mailto:alexove@fedoraproject.org}{\textbf{alexove@}\\fedoraproject.org}
    ~
    \section{Web \& Git}
    \href{https://github.com/alexove}{https://github.com/alexove}
    % ~
    % % use  \hspace{} or \vspace{} to change bubble size, if needed
    % \section{Cualidades Personales}
    % \smartdiagram[bubble diagram]{
    % \textbf{Trabajo}\\\textbf{en equipo},
    % \textbf{Iniciativa},
    % \textbf{Curiosidad},
    % \textbf{Resolucion}\\\textbf{problemas},
    % \textbf{\vspace{2mm}Gerencia\vspace{2mm}},
    % \textbf{Organizado}
    % }
    ~
    \section{Idiomas}
    \textbf{Espa\~nol}\includegraphics[scale=0.40]{img/5stars.png}
    \textbf{Ingles}\includegraphics[scale=0.40]{img/3stars.png}
    \textbf{Ruso}\includegraphics[scale=0.40]{img/2stars.png}
    ~
\end{aside}
~
\section{Ingeniero Inform\'atico y de Sistemas}
\emph{Con predisposici\'on para el cumplimiento de metas y objetivos, actitud pro-activa,
capacidad para trabajar en equipos multidisciplinarios, Trabajo en equipo bajo presi\'on,
capacidad de adaptaci\'on a cambios, alta confiabilidad, aspiraci\'on de desarrollo
profesional y capacidad para interactuar con personas dentro y fuera de la organizaci\'on y
deseos de progreso continuo.}
\\
\section{Educaci\'on}
\begin{entrylist}
    \entry
    {2019 - 2021}
    {Maestria en Seguridad Informatica}
    {Universidad de la Rioja - Sede Mexico}
    {Maestría enfocada en las \'ultimas t\'ecnicas de protecci\'on ante
    vulnerabilidades de sistemas operativos, software, bases de datos, sistemas
    web y todos los riesgos inherentes al empleo de las TIC y su posible repercusi\'on
    dentro de la estructura organizativa p\'ublica y privada.\\}
    \entry
    {2016 - 2017}
    {Maestria en Administraci\'on de negocios}
    {Universidad Andina del Cusco}
    {La Maestr\'ia en Administraci\'on de Negocios es un programa orientado a la formaci\'on
    profesionales de la Region Cusco y  del pa\'is, comprometi\'endolos con la realidad
    de un entorno que exige cada vez m\'as el conocimiento y el uso correcto y
    adecuado de la informaci\'on, con capacidad para responder con rigor, oportunidad
    y pertinencia social a los problemas locales, regionales y nacionales.\\}
    \entry
    {2009 - 2012}
    {Ing. Informatico y de Sistemas}
    {Universidad Nacional de San Antonio Abad del Cusco}
    {Profesional \'integro, \'etico, solidario, con s\'olida formaci\'on en la ciencias
    b\'asicas, Inform\'atica y de sistemas. Competente en el desarrollo de proyectos
    inform\'aticos, con capacidad para dise\~narlos, gerenciarlos, auditarlos y
    llevar el control y avance de los mismos.\\}
    \entry
    {2002 - 2009}
    {Bach. en Ing. Informatica y de Sistemas}
    {Universidad Nacional de San Antonio Abad del Cusco}
    {Profesional \'integro, \'etico, solidario, con s\'olida formaci\'on en la ciencias
    b\'asicas, Inform\'atica y de sistemas. Competente en el desarrollo de proyectos
    inform\'aticos, con capacidad para dise\~narlos, gerenciarlos, auditarlos y
    llevar el control y avance de los mismos.}
    \entry
    {1997 - 2001}
    {Educaci\'on Secundaria}
    {Colegio Particular Mixto Isaiah Bowman}
    {La Institucion Educativa Isaiah Bowman se creo el a\~no 1985, con 31 a\~nos de historia
    y miles de jovenes egresados de esta prestigiosa instituci\'on.}
\end{entrylist}
\\

\newpage
\begin{aside}
    ~
    ~
    \section{Lenguajes de programaci\'on}
    \includegraphics[scale=0.20]{img/java}
    \includegraphics[scale=0.07]{img/groovy}
    \includegraphics[scale=0.3]{img/bash}
    ~
    ~
    \section{Preferencia de Sistema Operativo}
    \textbf{GNU/Linux}\includegraphics[scale=0.40]{img/5stars.png}
    \textbf{Windows}\includegraphics[scale=0.40]{img/1stars.png}
    ~
    \section{Bases de datos}
    ~
    \includegraphics[scale=0.04]{img/mysql.png}
    \includegraphics[scale=0.30]{img/mariadb.png}
    ~
    \section{Desarrollo web}
    ~
    \includegraphics[scale=0.02]{img/grails.png}
    ~
\end{aside}
\section{Certificaciones}
\begin{entrylist}
    \entry
    {04/2016}
    {Linux Foundation Certified System Administrator LFCS}
    {Linux Foundation}
    {Linux Foundation certifications give you a way to differentiate yourself in
    a job market that's hungry for your skills. We've taken a new, innovative
    approach to Linux certification that allows you to showcase your skills in a
     way that other sysadmins will respect and employers will trust.\\
     \emph{Certificate ID N°{:} LFCS-1600-0744-0100}}
\end{entrylist}

\section{Experiencia}
\begin{entrylist}
    \entry
    {03/24 - Ahora}
    {Responsable de Data Center}
    {Gobierno Regional de Cusco}
    {Actividades realizadas: Responsable de Data Center (Administraci\'on de servidores GNU/Linux, supervisi\'on
    de redes), Encargado de soporte y mantenimiento de SIAF-MEF. Gobierno Digital y Seguridad de la Información.}
    \entry
    {09/23 - 02/24}
    {Desarrollador BIM}
    {Consorcio Rios del Norte S.A.C.}
    {Actividades realizadas: Planificación, desarrollo de sistma de gestión de calidad y automatización de procesos dentro 
    del marco del BIM.}
    \entry
    {12/22 - 08/23}
    {Jefe de Informatica y Sistemas}
    {MacSalud S.A.C.}
    {Actividades realizadas: Planificación, desarrollo, asignación de recursos y tareas, coordinación de proyectos de 
     sistemas de información, soporte de redes, centro de datos e Internet.}
    \entry
    {02/22 - 11/22}
    {Coordinador de proyecto de software}
    {Gerencia Regional de Educación del Cusco}
    {Actividades realizadas: Planificación, desarrollo, asignación de recursos y tareas,
     ejecución, seguimiento y entrega de Plataforma Virtual Educativa CREE.}
    \entry
    {08/21 - 09/21}
    {Analista de Seguridad de Información y Aplicaciones}
    {Caja Municipal Cusco}
    {Actividades realizadas: Monitoreo de controles de seguridad de la información. Analisis de Riesgos de Seguridad
    de la Información.}
    \entry
    {03/21 - 07/21}
    {Responsable de Data Center}
    {Gobierno Regional del Cusco}
    {Actividades realizadas: Responsable de Data Center (Administraci\'on de servidores GNU/Linux, supervisi\'on
    de redes), Encargado de soporte y mantenimiento de SIGA-MEF y SIAF.\\}
    \entry
    {01/20 - 03/21}
    {Responsable de la Oficina Funcional de Informatica}
    {Gobierno Regional del Cusco}
    {Actividades realizadas: Jefatura de la Oficina Funcional de Informatica, coordinación de proyectos de sistemas de información,
     soporte de redes, centro de datos e Internet. Parte del Comite de Gobierno Digital y Seguridad Digital.\\}
    \entry
    {01/18 - 12/19}
    {Responsable de Data Center}
    {Gobierno Regional del Cusco}
    {Actividades realizadas: Responsable de Data Center (Administraci\'on de servidores GNU/Linux, supervisi\'on
    de redes), Encargado de soporte y mantenimiento de SIGA-MEF y SIAF.\\}
    \entry
    {02/17 - 06/17}
    {Desarrollador de software GIS}
    {CEC Guaman Poma de Ayala}
    {Desarrollador de Sistema de Informaci\'on Georeferenciado utilizando java con gvSIG y PostgreSQL.\\}
\end{entrylist}
\begin{entrylist}
    \entry
    {04/16 - 07/16}
    {Responsable de Data Center}
    {Gobierno Regional del Cusco}
    {Actividades realizadas: Responsable de Data Center (Administraci\'on de servidores GNU/Linux, supervisi\'on
    de redes), Encargado de soporte y mantenimiento de SIGA-MEF y SIAF.\\}
    \entry
    {02/16 - 03/16}
    {Docente de Ofim\'atica}
    {CENFOTI – Universidad Andina del Cusco.}
    {Actividades realizadas: Docente de Ofim\'atica.\\}
    \entry
    {07/15 - 12/15}
    {Responsable de Data Center}
    {Gobierno Regional del Cusco}
    {Responsable de Data Center (Administraci\'on de servidores GNU/Linux, supervisi\'on de redes), Encargado de
    soporte y mantenimiento de SIGA-MEF.\\}
    \entry
    {01/15 - 03/15}
    {Responsable de Data Center}
    {Gobierno Regional del Cusco}
    {Responsable de Data Center (Administraci\'on de servidores GNU/Linux, supervisi\'on de redes), Encargado de
    soporte y mantenimiento de SIGA-MEF.\\}
    \entry
    {06/14 - 10/14}
    {Desarrollador de software}
    {Las Bambas - ManPower}
    {Desarrollador de aplicaciones web para mina (sistema de informaci\'on para perforaciones y voladuras).\\}
    \entry
    {07/12 - 02/14}
    {Desarrollador Web con Java}
    {Java For Smart Information Technologies - Cusco}
    {Desarrollador Web con Java, JSP. GWT b\'asico, Desarrollo de Android Intermedio, Framework GRAILS, Administraci\'on de Servidores virtuales.\\}
    \entry
    {11/12 - 01/13}
    {Responsable de Data Center}
    {{RootWay Internet Services \& Consulting - Uruguay}}
    {{Administraci\'on y migraci\'on de servidores virtuales, GNU/Linux en AmazonWS. \\ }}
    \entry
    {01/12 - 05/12}
    {Jefe de la Oficina de Inform\'atica.}
    {Municipalidad Provincial de Quispicanchi}
    {Jefe de la Oficina de Inform\'atica.\\}
\end{entrylist}
\begin{aside}
    ~
    \section{GNU/Linux}
    ~
    \includegraphics[scale=0.12]{img/fedora}
    \includegraphics[scale=0.12]{img/centos}
    ~
\end{aside}
\section {Diplomados}
\begin{entrylist}
  \entry
  {10/2020 - 03/2021}
  {Gestión de la Seguridad y Marco Legal}
  {Universidad Internacional de la Riona en México}
  {Realizado del 26 de octubre de 2020 al 7 de marzo de 2021, con una duración de 432 horas}
  \entry
  {05/2020 - 08/2020}
  {Seguridad en los Nuevos Entornos y Auditoría}
  {Universidad Internacional de la Riona en México}
  {Realizado del 4 de mayo de 2020 al 30 de agosto de 2020, con una duración de 360 horas}
  \entry
  {10/2019 - 03/2020}
  {Seguridad en Redes, Sistemas y Aplicaciones}
  {Universidad Internacional de la Riona en México}
  {Realizado del 28 de octubre de 2019 al 8 de marzo de 2020, con una duración de 360 horas}
\end{entrylist}
\newpage
\section{Cursos y seminarios}
\begin{entrylist}
    \entry
    {08/21 - 08/21}
    {Analisis de Vulnerabilidades Digitales}
    {Centro Nacional de Seguridad Digital}
    {Curso ofrecido por el Centro Nacional de Seguridad Digital (PCM-SEGDI) durante 24 horas lectivas.\\}
    \entry
    {06/21 - 06/21}
    {Auditor Interno ISO 27001}
    {Centro Nacional de Seguridad Digital}
    {Curso ofrecido por el Centro Nacional de Seguridad Digital (PCM-SEGDI) durante 24 horas lectivas.\\}
    \entry
    {05/21 - 05/21}
    {Linux Essentials}
    {Centro Nacional de Seguridad Digital}
    {Curso ofrecido por el Centro Nacional de Seguridad Digital (PCM-SEGDI) durante 24 horas lectivas.\\}
    \entry
    {12/15 - 01/16}
    {Curso OFICIAL ITIL FOUNDATION}
    {BSGrupo}
    {Realizado entre el 06 de diciembre del 2015 y el 06 de enero del 2016.\\}
\end{entrylist}

\section{Congresos y ponencias}
\begin{entrylist}
  \entry
    {11/2016}
    {\textbf{Ponente} VI Encuentro de Ing. de Sistemas e Inform\'atica }
    {UTEA}
    {Realizado del 16 al 18 de noviembre del 2015 en la Universidad Tecnológica de los Andes - Sede Cusco.\\}
\end{entrylist}
\begin{entrylist}
    \entry
    {09/2015}
    {\textbf{Ponente} Fedora Users and Developers Conference - Argentina}
    {Fedora Community}
    {Realizando en la ciudad de Cordoba-Argentina, entre 10 y 12 de septiembre
    del 2015 en la Universidad Nacional de C\'ordoba – Argentina.\\}
    \entry
    {08/2015}
    {\textbf{Ponente} {XXIII CONEISC}}
    {Universidad Catolica Santa Maria}
    {Realizado entre el 17 al 21 de agosto del 2015 en la Universidad Catolica Santa Maria, Arequipa, Peru.\\}
    \entry
    {11/2014}
    {\textbf{Ponente} VI Festival Internacional de Software Libre SOFLISEOANE}
    {Instituto de Educaci\'on Superior Tecnol\'ogico Publico Manuel Seoane Corrales}
    {Realizado el 28 y 29 de noviembre del 2014 en el Instituto de Educaci\'on Superior
    Tecnol\'ogico Publico Manuel Seoane Corrales, Lima.\\}
    \entry
    {04/2014}
    {\textbf{Ponente} 15-vo Foro Internacional do Software Livre-Porto Alegre-Brasil}
    {FISL}
    {Realizado entre el 7 y 10 de mayo del 2014 en la ciudad de Porto Alegre-Brasil.\\}
    \entry
    {01/14 - 03/14}
    {Curso Taller de "SEGURIDAD INFORM\'ATICA"}
    {Grupo Development In Peru SCRL}
    {Organizado por el Grupo Development In Peru SCRL.\\}
    \entry
    {12/2015}
    {\textbf{Ponente} Semana Informatica INSANE CODE}
    {IST Tupac Amaru}
    {Organizado por el Instituto de Educaci\'on Superior Tecnol\'ogico Publico "Tupac
    Amaru" del Cusco.}
\end{entrylist}
\begin{entrylist}
    \entry
    {10-2013}
    {\textbf{Ponente} IV ENCUENTRO DE INGENIERIA DE SISTEMAS E INFORMATICA}
    {UTEA}
    {Realizado los d\'ias 26 y 27 de octubre del 2013 organizado por la Universidad
    Tecnol\'ogica de los Andes.\\}
    \entry
    {01/13 - 03/13}
    {Curso ANDROID{:} PROGRAMACI\'ON DE APLICACIONES}
    {UPV}
    {Realizado del 10 de enero del 2013 hasta el 30 de marzo del 2013, dado por
    la Universidad Polit\'ecnica de Valencia-Espa\~na.}
    \entry
    {09-2012}
    {Encuentro GOOGLE DEVELOPMENT FEST LIMA}
    {Google Developers Group}
    {Realizado el 27 y 28 de octubre del 2012 y organizado por Google Developer Group Lima.},
    \entry
    {08-2012}
    {Seminario Taller de Regional ESTRATEGIA NACIONAL DE GOBIERNO ELECTRONICO}
    {ONGEI}
    {Realizado el 13 y 14 de septiembre del 2012, organizado por la Oficina
    Nacional de Gobierno Electr\'onico e Inform\'atica.}
\end{entrylist}
\section{Otras actividades}
\begin{entrylist}
    \entry
    {08-2013}
    {Organizador de FUDCon Cusco 2013}
    {Fedora Communitty, UNSAAC}
    {Fedora Users and Developers Conference Latin América Cusco 2013, organizado
    por la comunidad Fedora Perú, Universidad Nacional de San Antonio Abad del
    Cusco y Municipalidad Provincial del Cusco, realizado el 26,27 y 28 de
    septiembre del 2013.}
    \entry
    {03/13 - ahora}
    {Embajador Fedora}
    {Fedora Community}
    {Ambassadors are the representatives of Fedora. Ambassadors ensure the public
    understand Fedora's principles and the work that Fedora is doing. Additionally
    Ambassadors are responsible for helping to grow the contributor base, and to
    act as a liaison between other FLOSS projects and the Fedora community. }
\end{entrylist}
%
%\begin{flushleft}
%    \emph{May 8th, 2016}
%\end{flushleft}
%\begin{flushright}
%    \emph{John Snow}
%\end{flushright}

\end{document}
